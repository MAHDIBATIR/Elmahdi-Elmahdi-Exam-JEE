\subsection{Test de la couche DAO}
Pour tester la couche DAO, nous avons créé une classe de test qui alimente la base de données avec des données de test :
\begin{lstlisting}[caption=Application de test de la couche DAO]
package ma.enset.elbatir_elmahdi;

import ma.enset.elbatir_elmahdi.entities.*;
import ma.enset.elbatir_elmahdi.repositories.*;
import org.springframework.boot.CommandLineRunner;
import org.springframework.boot.SpringApplication;
import org.springframework.boot.autoconfigure.SpringBootApplication;
import org.springframework.context.annotation.Bean;
import java.util.Date;
import java.util.stream.Stream;

@SpringBootApplication
public class CreditBancaireApplication {
    public static void main(String[] args) {
        SpringApplication.run(CreditBancaireApplication.class, args);
    }

    @Bean
    CommandLineRunner start(ClientRepository clientRepository,
                           CreditRepository creditRepository,
                           CreditPersonnelRepository creditPersonnelRepository,
                           CreditImmobilierRepository creditImmobilierRepository,
                           CreditProfessionnelRepository creditProfessionnelRepository,
                           RemboursementRepository remboursementRepository,
                           UserRepository userRepository,
                           RoleRepository roleRepository) {
        return args -> {
            // Création des clients
            Stream.of("Client1", "Client2", "Client3").forEach(name -> {
                Client client = new Client();
                client.setName(name);
                client.setEmail(name.toLowerCase() + "@gmail.com");
                clientRepository.save(client);
            });

            // Création des crédits personnels
            Client client1 = clientRepository.findById(1L).orElse(null);
            CreditPersonnel creditPersonnel = new CreditPersonnel();
            creditPersonnel.setClient(client1);
            creditPersonnel.setRequestDate(new Date());
            creditPersonnel.setStatus(CreditStatus.EN_COURS);
            creditPersonnel.setAmount(50000.0);
            creditPersonnel.setDuration(24);
            creditPersonnel.setInterestRate(4.5);
            creditPersonnel.setMotif("Achat de voiture");
            creditPersonnelRepository.save(creditPersonnel);

            // Création des crédits immobiliers
            Client client2 = clientRepository.findById(2L).orElse(null);
            CreditImmobilier creditImmobilier = new CreditImmobilier();
            creditImmobilier.setClient(client2);
            creditImmobilier.setRequestDate(new Date());
            creditImmobilier.setStatus(CreditStatus.ACCEPTE);
            creditImmobilier.setAcceptanceDate(new Date());
            creditImmobilier.setAmount(200000.0);
            creditImmobilier.setDuration(240);
            creditImmobilier.setInterestRate(2.5);
            creditImmobilier.setPropertyType(PropertyType.APPARTEMENT);
            creditImmobilierRepository.save(creditImmobilier);

            // Création des crédits professionnels
            Client client3 = clientRepository.findById(3L).orElse(null);
            CreditProfessionnel creditProfessionnel = new CreditProfessionnel();
            creditProfessionnel.setClient(client3);
            creditProfessionnel.setRequestDate(new Date());
            creditProfessionnel.setStatus(CreditStatus.EN_COURS);
            creditProfessionnel.setAmount(100000.0);
            creditProfessionnel.setDuration(48);
            creditProfessionnel.setInterestRate(3.5);
            creditProfessionnel.setMotif("Acquisition de matériel");
            creditProfessionnel.setRaisonSociale("Entreprise ABC");
            creditProfessionnelRepository.save(creditProfessionnel);

            // Création des remboursements
            Credit credit1 = creditRepository.findById(1L).orElse(null);
            Remboursement remboursement1 = new Remboursement();
            remboursement1.setCredit(credit1);
            remboursement1.setDate(new Date());
            remboursement1.setAmount(2000.0);
            remboursement1.setType(RemboursementType.MENSUALITE);
            remboursementRepository.save(remboursement1);

            // Création des rôles
            Role roleAdmin = new Role();
            roleAdmin.setRoleName("ROLE_ADMIN");
            roleRepository.save(roleAdmin);

            Role roleEmploye = new Role();
            roleEmploye.setRoleName("ROLE_EMPLOYE");
            roleRepository.save(roleEmploye);

            Role roleClient = new Role();
            roleClient.setRoleName("ROLE_CLIENT");
            roleRepository.save(roleClient);

            // Création des utilisateurs
            User admin = new User();
            admin.setUsername("admin");
            admin.setPassword("$2a$10$iyH.Uiv1Rx67gBdEXIabqOHPzxBsfpjmC0zM9JMs6i4tU0ymvZZie"); // password: admin
            admin.getRoles().add(roleAdmin);
            userRepository.save(admin);

            User employe = new User();
            employe.setUsername("employe");
            employe.setPassword("$2a$10$10.lLyY6aB.jxTG9l.8HzeRwX5QmP.IQDMNi4AcaX3mQFxVJYIUYy"); // password: employe
            employe.getRoles().add(roleEmploye);
            userRepository.save(employe);

            User clientUser = new User();
            clientUser.setUsername("client");
            clientUser.setPassword("$2a$10$RlU/kbVqHPKN2hD5xJXcGOZ9.MTUxNRy4yXWxjOiQW3/AEf1YybQW"); // password: client
            clientUser.getRoles().add(roleClient);
            userRepository.save(clientUser);
        };
    }
}
\end{lstlisting}

\subsection{Validation de la Base de Données}
Après l'exécution des tests de la couche DAO, nous avons validé la création et le peuplement des tables dans la base de données. Voici les captures d'écran des tables créées :

\subsubsection{Table Client}
\begin{figure}[H]
    \centering
    \includegraphics[width=0.8\textwidth]{client_table.png}
    \caption{Structure et données de la table Client}
\end{figure}

\subsubsection{Table Credit}
\begin{figure}[H]
    \centering
    \includegraphics[width=0.8\textwidth]{credit_table.png}
    \caption{Structure et données de la table Credit}
\end{figure}

\subsubsection{Table CreditPersonnel}
\begin{figure}[H]
    \centering
    \includegraphics[width=0.8\textwidth]{credit_personnel_table.png}
    \caption{Structure et données de la table CreditPersonnel}
\end{figure}

\subsubsection{Table CreditImmobilier}
\begin{figure}[H]
    \centering
    \includegraphics[width=0.8\textwidth]{credit_immobilier_table.png}
    \caption{Structure et données de la table CreditImmobilier}
\end{figure}

\subsubsection{Table CreditProfessionnel}
\begin{figure}[H]
    \centering
    \includegraphics[width=0.8\textwidth]{credit_professionnel_table.png}
    \caption{Structure et données de la table CreditProfessionnel}
\end{figure}

\subsubsection{Table Remboursement}
\begin{figure}[H]
    \centering
    \includegraphics[width=0.8\textwidth]{remboursement_table.png}
    \caption{Structure et données de la table Remboursement}
\end{figure}

\subsubsection{Table User et Role}
\begin{figure}[H]
    \centering
    \includegraphics[width=0.8\textwidth]{user_role_tables.png}
    \caption{Structure et données des tables User et Role}
\end{figure}

Ces captures d'écran montrent que :
\begin{itemize}
    \item Les tables ont été correctement créées avec leurs relations
    \item Les données de test ont été insérées avec succès
    \item Les contraintes de clés étrangères sont respectées
    \item Les types de données sont appropriés pour chaque colonne
\end{itemize}

\section{Couche Service}
\subsection{Création des DTOs}
Les DTOs (Data Transfer Objects) sont utilisés pour transférer les données entre la couche service et la couche web. Voici l'implémentation des DTOs :

\begin{lstlisting}[caption=DTO pour Client]
package ma.enset.elbatir_elmahdi.dtos;

import lombok.AllArgsConstructor;
import lombok.Data;
import lombok.NoArgsConstructor;

@Data
@NoArgsConstructor
@AllArgsConstructor
public class ClientDTO {
    private Long id;
    private String name;
    private String email;
}
\end{lstlisting}

\begin{lstlisting}[caption=DTO pour Credit]
package ma.enset.elbatir_elmahdi.dtos;

import lombok.AllArgsConstructor;
import lombok.Data;
import lombok.NoArgsConstructor;
import ma.enset.elbatir_elmahdi.entities.CreditStatus;

import java.util.Date;

@Data
@NoArgsConstructor
@AllArgsConstructor
public class CreditDTO {
    private Long id;
    private Date requestDate;
    private CreditStatus status;
    private Date acceptanceDate;
    private Double amount;
    private Integer duration; // Duration in months
    private Double interestRate;
    private Long clientId;
}
\end{lstlisting}

\subsection{Mappers}
Les mappers sont utilisés pour convertir les entités en DTOs et vice versa. Voici un exemple d'implémentation de mapper :

\begin{lstlisting}[caption=Mapper pour Client]
package ma.enset.elbatir_elmahdi.mappers;

import ma.enset.elbatir_elmahdi.dtos.ClientDTO;
import ma.enset.elbatir_elmahdi.entities.Client;
import org.springframework.stereotype.Component;

@Component
public class ClientMapper {
    public ClientDTO fromClient(Client client) {
        if (client == null) return null;
        return new ClientDTO(
                client.getId(),
                client.getName(),
                client.getEmail()
        );
    }

    public Client fromClientDTO(ClientDTO clientDTO) {
        if (clientDTO == null) return null;
        Client client = new Client();
        client.setId(clientDTO.getId());
        client.setName(clientDTO.getName());
        client.setEmail(clientDTO.getEmail());
        return client;
    }
}
\end{lstlisting}

\subsection{Services}
Les services définissent l'interface des opérations métier. Ils font le lien entre les contrôleurs et les repositories. Voici l'implémentation de l'interface de service pour les clients :

\begin{lstlisting}[caption=Interface Service pour Client]
package ma.enset.elbatir_elmahdi.services;

import ma.enset.elbatir_elmahdi.dtos.ClientDTO;

import java.util.List;

public interface ClientService {
    
    // CRUD operations
    ClientDTO saveClient(ClientDTO clientDTO);
    ClientDTO getClient(Long id);
    List<ClientDTO> getAllClients();
    ClientDTO updateClient(ClientDTO clientDTO);
    void deleteClient(Long id);
    
    // Additional operations
    ClientDTO getClientByEmail(String email);
}
\end{lstlisting}

\begin{lstlisting}[caption=Implémentation du Service Client]
package ma.enset.elbatir_elmahdi.services.impl;

import lombok.RequiredArgsConstructor;
import ma.enset.elbatir_elmahdi.dtos.ClientDTO;
import ma.enset.elbatir_elmahdi.entities.Client;
import ma.enset.elbatir_elmahdi.mappers.ClientMapper;
import ma.enset.elbatir_elmahdi.repositories.ClientRepository;
import ma.enset.elbatir_elmahdi.services.ClientService;
import org.springframework.stereotype.Service;
import org.springframework.transaction.annotation.Transactional;

import java.util.List;
import java.util.stream.Collectors;

@Service
@Transactional
@RequiredArgsConstructor
public class ClientServiceImpl implements ClientService {
    private final ClientRepository clientRepository;
    private final ClientMapper clientMapper;

    @Override
    public ClientDTO saveClient(ClientDTO clientDTO) {
        Client client = clientMapper.fromClientDTO(clientDTO);
        Client savedClient = clientRepository.save(client);
        return clientMapper.fromClient(savedClient);
    }

    @Override
    public ClientDTO getClient(Long id) {
        Client client = clientRepository.findById(id).orElse(null);
        return clientMapper.fromClient(client);
    }

    @Override
    public List<ClientDTO> getAllClients() {
        List<Client> clients = clientRepository.findAll();
        return clients.stream()
                .map(clientMapper::fromClient)
                .collect(Collectors.toList());
    }

    @Override
    public ClientDTO updateClient(ClientDTO clientDTO) {
        Client client = clientMapper.fromClientDTO(clientDTO);
        Client updatedClient = clientRepository.save(client);
        return clientMapper.fromClient(updatedClient);
    }

    @Override
    public void deleteClient(Long id) {
        clientRepository.deleteById(id);
    }

    @Override
    public ClientDTO getClientByEmail(String email) {
        Client client = clientRepository.findByEmail(email);
        return clientMapper.fromClient(client);
    }
}
\end{lstlisting}