\section{Couche Web}
\subsection{Contrôleurs REST}
Les contrôleurs REST exposent les fonctionnalités métier sous forme d'API RESTful. Voici l'implémentation du contrôleur pour les clients :

\begin{lstlisting}[caption=Contrôleur REST pour Client]
package ma.enset.elbatir_elmahdi.web;

import lombok.RequiredArgsConstructor;
import ma.enset.elbatir_elmahdi.dtos.ClientDTO;
import ma.enset.elbatir_elmahdi.services.ClientService;
import org.springframework.http.ResponseEntity;
import org.springframework.web.bind.annotation.*;

import java.util.List;

@RestController
@RequestMapping("/api/clients")
@RequiredArgsConstructor
@CrossOrigin("*")
public class ClientRestController {
    private final ClientService clientService;

    @GetMapping
    public List<ClientDTO> getAllClients() {
        return clientService.getAllClients();
    }

    @GetMapping("/{id}")
    public ResponseEntity<ClientDTO> getClientById(@PathVariable Long id) {
        ClientDTO clientDTO = clientService.getClient(id);
        if (clientDTO == null) {
            return ResponseEntity.notFound().build();
        }
        return ResponseEntity.ok(clientDTO);
    }

    @GetMapping("/email/{email}")
    public ResponseEntity<ClientDTO> getClientByEmail(@PathVariable String email) {
        ClientDTO clientDTO = clientService.getClientByEmail(email);
        if (clientDTO == null) {
            return ResponseEntity.notFound().build();
        }
        return ResponseEntity.ok(clientDTO);
    }

    @PostMapping
    public ClientDTO createClient(@RequestBody ClientDTO clientDTO) {
        return clientService.saveClient(clientDTO);
    }

    @PutMapping("/{id}")
    public ResponseEntity<ClientDTO> updateClient(@PathVariable Long id, @RequestBody ClientDTO clientDTO) {
        clientDTO.setId(id);
        ClientDTO updatedClient = clientService.updateClient(clientDTO);
        return ResponseEntity.ok(updatedClient);
    }

    @DeleteMapping("/{id}")
    public ResponseEntity<Void> deleteClient(@PathVariable Long id) {
        clientService.deleteClient(id);
        return ResponseEntity.noContent().build();
    }
}
\end{lstlisting}

\begin{lstlisting}[caption=Contrôleur REST pour Credit]
package ma.enset.elbatir_elmahdi.web;

import lombok.RequiredArgsConstructor;
import ma.enset.elbatir_elmahdi.dtos.CreditDTO;
import ma.enset.elbatir_elmahdi.services.CreditService;
import org.springframework.http.ResponseEntity;
import org.springframework.web.bind.annotation.*;

import java.util.List;

@RestController
@RequestMapping("/api/credits")
@RequiredArgsConstructor
@CrossOrigin("*")
public class CreditRestController {
    private final CreditService creditService;

    @GetMapping
    public List<CreditDTO> getAllCredits() {
        return creditService.getAllCredits();
    }

    @GetMapping("/{id}")
    public ResponseEntity<CreditDTO> getCreditById(@PathVariable Long id) {
        CreditDTO creditDTO = creditService.getCredit(id);
        if (creditDTO == null) {
            return ResponseEntity.notFound().build();
        }
        return ResponseEntity.ok(creditDTO);
    }

    @GetMapping("/client/{clientId}")
    public List<CreditDTO> getCreditsByClientId(@PathVariable Long clientId) {
        return creditService.getCreditsByClientId(clientId);
    }

    @DeleteMapping("/{id}")
    public ResponseEntity<Void> deleteCredit(@PathVariable Long id) {
        creditService.deleteCredit(id);
        return ResponseEntity.noContent().build();
    }
}
\end{lstlisting}

\subsection{Documentation Swagger}
La documentation des API est générée avec Springdoc OpenAPI. Voici la configuration utilisée :

\begin{lstlisting}[caption=Configuration Swagger via OpenApiConfig]
package ma.enset.elbatir_elmahdi.config;

import io.swagger.v3.oas.models.Components;
import io.swagger.v3.oas.models.OpenAPI;
import io.swagger.v3.oas.models.info.Contact;
import io.swagger.v3.oas.models.info.Info;
import io.swagger.v3.oas.models.security.SecurityRequirement;
import io.swagger.v3.oas.models.security.SecurityScheme;
import org.springframework.context.annotation.Bean;
import org.springframework.context.annotation.Configuration;

@Configuration
public class OpenApiConfig {

    @Bean
    public OpenAPI customOpenAPI() {
        return new OpenAPI()
                .info(new Info()
                        .title("API de Gestion des Crédits Bancaires")
                        .version("1.0")
                        .description("Cette API permet de gérer les clients, crédits et remboursements")
                        .contact(new Contact()
                                .name("EL BATIR EL MAHDI")
                                .email("e.elbatir@gmail.com")))
                .components(new Components()
                        .addSecuritySchemes("bearer-jwt",
                                new SecurityScheme()
                                        .type(SecurityScheme.Type.HTTP)
                                        .scheme("bearer")
                                        .bearerFormat("JWT")
                                        .in(SecurityScheme.In.HEADER)
                                        .name("Authorization")))
                .addSecurityItem(new SecurityRequirement().addList("bearer-jwt"));
    }
}
\end{lstlisting}

\section{Sécurité}
\subsection{Configuration Spring Security}
La sécurité est basée sur Spring Security. La configuration actuelle est configurée pour permettre le développement, mais prête à être remplacée par JWT :

\begin{lstlisting}[caption=Configuration de Sécurité]
package ma.enset.elbatir_elmahdi.config;

import org.springframework.context.annotation.Bean;
import org.springframework.context.annotation.Configuration;
import org.springframework.security.config.annotation.web.builders.HttpSecurity;
import org.springframework.security.config.annotation.web.configuration.EnableWebSecurity;
import org.springframework.security.config.annotation.web.configurers.HeadersConfigurer;
import org.springframework.security.web.SecurityFilterChain;
import org.springframework.web.cors.CorsConfiguration;
import org.springframework.web.cors.CorsConfigurationSource;
import org.springframework.web.cors.UrlBasedCorsConfigurationSource;

import java.util.Arrays;

@Configuration
@EnableWebSecurity
public class SecurityConfig {

    @Bean
    public SecurityFilterChain securityFilterChain(HttpSecurity http) throws Exception {
        // This is a temporary configuration for development and testing
        // We will replace it with JWT security in question 6
        http
                .csrf(csrf -> csrf.disable())
                .cors(cors -> cors.configurationSource(corsConfigurationSource()))
                .headers(headers -> headers.frameOptions(HeadersConfigurer.FrameOptionsConfig::disable))
                .authorizeHttpRequests(auth -> auth
                        .anyRequest().permitAll() // Temporarily allow all requests
                );
        return http.build();
    }

    @Bean
    CorsConfigurationSource corsConfigurationSource() {
        CorsConfiguration configuration = new CorsConfiguration();
        configuration.setAllowedOrigins(Arrays.asList("http://localhost:4200"));
        configuration.setAllowedMethods(Arrays.asList("HEAD", "GET", "POST", "PUT", "DELETE", "PATCH", "OPTIONS"));
        configuration.setAllowedHeaders(Arrays.asList("Authorization", "Cache-Control", "Content-Type"));
        configuration.setAllowCredentials(true);
        UrlBasedCorsConfigurationSource source = new UrlBasedCorsConfigurationSource();
        source.registerCorsConfiguration("/**", configuration);
        return source;
    }
}
\end{lstlisting}

\section{Frontend Angular}
\subsection{Structure du Frontend}
Le frontend Angular est structuré en composants, services et modèles. La structure est la suivante :

\begin{lstlisting}[caption=Structure du frontend Angular]
frontend/
├── src/
│   ├── app/
│   │   ├── components/
│   │   │   ├── clients/
│   │   │   │   ├── client-list/
│   │   │   │   ├── client-form/
│   │   │   │   └── client-details/
│   │   │   ├── credits/
│   │   │   │   ├── credit-list/
│   │   │   │   ├── credit-form/
│   │   │   │   └── credit-details/
│   │   │   └── remboursements/
│   │   ├── models/
│   │   │   ├── client.model.ts
│   │   │   ├── credit.model.ts
│   │   │   └── remboursement.model.ts
│   │   ├── services/
│   │   │   ├── client.service.ts
│   │   │   ├── credit.service.ts
│   │   │   └── remboursement.service.ts
│   │   ├── app.component.ts
│   │   ├── app.module.ts
│   │   └── app-routing.module.ts
│   ├── assets/
│   └── index.html
└── package.json
\end{lstlisting}

\subsection{Services Angular}
Les services Angular permettent de communiquer avec le backend via des requêtes HTTP. Voici un exemple de service pour gérer les clients :

\begin{lstlisting}[caption=Service Angular pour Client]
import { Injectable } from '@angular/core';
import { HttpClient } from '@angular/common/http';
import { Observable } from 'rxjs';
import { Client } from '../models/client.model';

@Injectable({
  providedIn: 'root'
})
export class ClientService {
  private apiUrl = 'http://localhost:8080/api/clients';

  constructor(private http: HttpClient) { }

  getAllClients(): Observable<Client[]> {
    return this.http.get<Client[]>(this.apiUrl);
  }

  getClientById(id: number): Observable<Client> {
    return this.http.get<Client>(`${this.apiUrl}/${id}`);
  }

  createClient(client: Client): Observable<Client> {
    return this.http.post<Client>(this.apiUrl, client);
  }

  updateClient(id: number, client: Client): Observable<Client> {
    return this.http.put<Client>(`${this.apiUrl}/${id}`, client);
  }

  deleteClient(id: number): Observable<void> {
    return this.http.delete<void>(`${this.apiUrl}/${id}`);
  }
}
\end{lstlisting}

\subsection{Composants Angular}
Les composants Angular représentent l'interface utilisateur. Voici un exemple de composant pour afficher la liste des clients :

\begin{lstlisting}[caption=Composant Angular pour la liste des clients]
import { Component, OnInit } from '@angular/core';
import { ClientService } from '../../services/client.service';
import { Client } from '../../models/client.model';

@Component({
  selector: 'app-client-list',
  templateUrl: './client-list.component.html',
  styleUrls: ['./client-list.component.css']
})
export class ClientListComponent implements OnInit {
  clients: Client[] = [];
  loading = false;
  error: string | null = null;

  constructor(private clientService: ClientService) { }

  ngOnInit(): void {
    this.loadClients();
  }

  loadClients(): void {
    this.loading = true;
    this.clientService.getAllClients()
      .subscribe({
        next: (data) => {
          this.clients = data;
          this.loading = false;
        },
        error: (err) => {
          this.error = 'Erreur lors du chargement des clients';
          this.loading = false;
          console.error(err);
        }
      });
  }

  deleteClient(id: number): void {
    if (confirm('Êtes-vous sûr de vouloir supprimer ce client?')) {
      this.clientService.deleteClient(id)
        .subscribe({
          next: () => {
            this.clients = this.clients.filter(client => client.id !== id);
          },
          error: (err) => {
            this.error = 'Erreur lors de la suppression du client';
            console.error(err);
          }
        });
    }
  }
}
\end{lstlisting}

\section{Améliorations Additionnelles}
Les améliorations suivantes ont été apportées au projet :

\begin{itemize}
    \item \textbf{Validation} : Ajout de contraintes de validation sur les entités et DTOs avec les annotations comme @NotNull, @Email, etc.
    \item \textbf{Gestion des exceptions} : Mise en place d'un gestionnaire global des exceptions pour traiter les erreurs de manière cohérente.
    \item \textbf{Documentation} : Documentation complète des API avec Swagger (OpenAPI).
    \item \textbf{Tests unitaires} : Implémentation de tests unitaires pour les services.
    \item \textbf{Interface utilisateur réactive} : Formulaires réactifs dans Angular pour une meilleure expérience utilisateur.
    \item \textbf{Sécurité renforcée} : Préparation pour l'implémentation de JWT pour l'authentification et l'autorisation.
\end{itemize}

\section{Conclusion}
Ce projet a permis de concevoir et développer une application JEE complète pour la gestion des crédits bancaires. L'utilisation de Spring Boot pour le backend et Angular pour le frontend offre une architecture moderne, évolutive et maintenable.

L'application répond aux exigences demandées en termes de fonctionnalités et d'architecture. Elle permet la gestion des clients, des différents types de crédits et des remboursements, avec une interface utilisateur conviviale et une sécurité renforcée.

Les perspectives d'amélioration pourraient inclure :
\begin{itemize}
    \item L'ajout de tableaux de bord analytiques pour visualiser les données des crédits
    \item L'implémentation de notifications en temps réel pour les changements de statut des crédits
    \item L'intégration avec des services externes comme des systèmes de scoring de crédit
    \item L'ajout d'une fonctionnalité de génération de rapports PDF
\end{itemize}

Le développement a été effectué en suivant les bonnes pratiques de programmation et en respectant les principes SOLID, ce qui rend l'application facilement maintenable et extensible pour des développements futurs.
